\documentclass[10pt,a4paper]{article}
\usepackage[utf8]{inputenc}
\usepackage[english]{babel}
\usepackage{amsmath}
\usepackage{amsfonts}
\usepackage{amssymb}
\usepackage{parskip}

%\addtolength{\topmargin}{-2cm}
\addtolength{\oddsidemargin}{-.5in}
\addtolength{\evensidemargin}{-.5in}
\addtolength{\textwidth}{1in}
%\addtolength{\textheight}{0.5cm}

\author{F.P. Avis \quad Student number 10904581 \quad Universiteit van Amsterdam}
\title{Self-study report Introduction to Computational Science}
\begin{document}
\maketitle

In this document I will explain my experiences in studying for the course Introduction to Computational Science.

My main background lies in Econometrics and Operations Research, with emphasis on the last. Consequently, I have a strong mathematical foundation and firm modelling experience. At the moment I'm following the High Performance Distributed Computing track of the master Computer Science at the Vrije Universiteit Amsterdam. Although it was good for me to refresh the mathematics in the seminars and enjoyed them, I highly doubt the relevance of this course for my current master programme. The bachelor in Computer Science at the VU does not discuss the mathematics used in this course.

I attended the seminars about differential equations and linear algebra. These subjects were discussed in my bachelor, but a refreshment was very good for me and I'm glad I had the opportunity to attended these seminars. It encouraged me to reread a large part of my book on linear algebra and this led to many new insights I didn't had at the moment I passed the exam. For the other seminars, I prepared the night before and was able to make all exercises properly and did decide to not attend these seminars.

The discussed material at the lectures was completely new for me. I have written my experiences with the lectures on the evaluation form at the exam and I will not repeat them here.

In my bachelor and master Econometrics and Operations Research I have obtained experience with linear programming models, time series, discrete event simulation and Markov decision models and additionally followed courses about combinatorial optimisation and evolutionary computing. I also have experience in Java and Matlab. This is why I wanted something new for the project. My partner (Marc Went) and I decided to implement the World3 model of the Club of Rome, discussed in the book Limits to Growth. This model heavily relies on differential equations and we decided to implement it in Python. I was not familiar with models based on differential equations, nor with Python. To get familiar with Python, I did some courses at http://www.codecademy.com, but learned that I prefer reading a book to get familiar with a new programming language. Eventually, Marc did most of the programming work in Python, from which I learned a lot, and created the presentation. I performed a literature study, wrote the most of the report and delved into the model to find the error in our program. I also learned how to use GitHub.

Unfortunately we did not succeed in implementing the model. We took a bit of a risk with the model, but I would haven't learned anything by staying on safe paths.

As a side note, we are already from a generation who grew up in the digital age. By choosing the World3 model, we had to borrow six books from the libaries at the UvA and VU. Reading scientific books for a project was a novelty for us.

\end{document}